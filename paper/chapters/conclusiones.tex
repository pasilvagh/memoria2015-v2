\chapter{Conclusiones}
\label{chap7:concl}

Un navegador Web pareciera ser un Software de mediana complejidad para tanto usuarios como desarrolladores sin experiencia en seguridad, pero lamentablemente esta pieza de Software permite realizar una variadad de vectores de ataque, tanto en un usuario usándolo como en el sistema con el que interactúa. Por lo tanto es importante comprender su estructura y como éste interactúa con Stakeholders internos como externos.


\section{Contribuciones}
Durante el curso de este trabajo se realizaron las siguientes contribuciones:
\begin{itemize}
	\item Una base conceptual para aquellos desarrolladores que no comprendan sobre términos de seguridad relacionados al navegador Web.
	\item Construcción del primer Patrón Arquitectural sobre la infraestructura del \textit{Web Browser}, para poder entender de manera holística los componentes, interacciones y relaciones.
	\item Una parte de la Arquitectura de Referencia ha sido construída, a través de la abstracción del Patrón Browser Infrastructure. Además se ha conseguido caracterizar los Stakeholders y Casos de Uso más importantes. De lo que tenemos por conocido, esta es la segunda Arquitectura de Referencia construída del \textit{Browser}. Sin embargo, nuestra propuesta es la más actual y se ajusta más al tipo de arquitectura que ahora los Browsers utilizan, usando una arquitectura Modular.
	\item Construcción de un Patrón de Mal Uso, como primera instancia para entender los conceptos de seguridad, como amenazas y ataques posibles de realizar dentro del navegador Web.
\end{itemize}

\section{Resumen}
El estudio de la seguridad del \textit{Web Browser} y la construcción de una Arquitectura de Referencia mediante patrones, conduce a las siguientes conclusiones:
\begin{itemize}
	\item Una śintesis y abstracción de la información correspondiente a los Web Browsers, para generar un lenguaje de comunicación de los conceptos. El uso de patrones permite ayudar a un conjunto de personas a unificar conceptos, así como también puede ser guía en futuros desarrollos.
	\item El trabajo propuesto permite comprender mejor, por medio de la Arquitectura de Referencia parcialmente construída, tanto componentes como amenazas existentes. Además como no está sujeto a implementaciones específicas, es posible generalizar ciertos resultados en otros Browsers.
	\item Comunicar en primera instancia, los conceptos de seguridad básicos para un mejor entendimiento de los subsistemas que interactuan en la Internet. Los patrones de Mal Uso permiten condensar la información y transmitirla, de forma ordenada y descriptiva.
	\item Es de esperarse que en el futuro la mayoría de los \textit{Web Browser} tomen forma de una Arquitectura Modular. Por lo tanto, es importante que los desarrolladores conozcan los procesos internos del \textit{Web Browser} al momento de desarrollar un sistema que se comunicará con éste. Tanto la Arquitectura de Referencia como el patrón de Mal Uso presentados, apuntan a entregar el conocimiento básico de los componentes e interacciones entre el \textit{Web Browser} y un proveedor de recursos externo; así como también de las amenazas que existen.
	
\end{itemize}


\section{Trabajo Futuro}
El trabajo futuro que se realizará para obtener el grado de Magister, irá relacionado a la creación de una Arquitectura de Referencia de seguridad del \textit{Web Browser}, utilizando la misma metodología presentada acá. Otros patrones relacionados al Patrón Browser Infrastructure serán obtenidos para así completar la AR ya iniciada, como por ejemplo el patrón Web Content Renderer y Browser Kernel. Un ejemplo del tipo de trabajo que se pretende realizar puede ser vista en \cite{fernandez2014security}, donde este estudio realiza actividades para poder construir software seguro y evaluar los niveles de seguridad de un sistema ya construído.

Se planea construir más Patrones de Mal Uso, para el Patrón Browser Infrastructure para continuar con el estudio de amenazas posibles dentro del \textit{Browser}, como una manera de educar a los Desarrolladores y Stakeholders de los peligros existentes. Al mismo tiempo estos patrones permitirán la construcción de esta AR de seguridad. En esta misma dirección, además de encontrar las amenazas posibles de existir en el sistema, se necesita encontrar las contramedidas o defensas de seguridad que permitan evitar o preveer esas amenazas a través de Patrones de Seguridad sobre la Arquitectura de Referencia construída. Lo anterior es posible de realizar bajo el mismo ejercicio ya realizado en este trabajo, buscando amenazas sobre cada acción realizada en cada Caso de Uso del navegador.

En cuanto a los \textit{Web Browser}, los ataques basados en Ingeniería Social parece que no disminuirán en un buen tiempo, pues no existen tecnologías actuales que puedan detectar en un \(100\%\) y sin falsos positivos los posibles peligros que pueden traer. Tecnologías como CAMP (Content-Agnostic Malware Protection) parecen ser parte de la solución, pero aún están lejos de ser perfectos.