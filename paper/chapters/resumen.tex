% Resumen a grandes rasgos de lo existente (resumen del estado del arte y resumen del trabajo realizado)

\section*{Resumen}
\label{chap:resumen}

El Web Browser es una de las aplicaciones más usadas - \textit{killer app} - y también una de las primeras en aparecer en cuanto se creó el Internet (Década de los 90). Por lo mismo, su nivel de madurez con respecto a otros desarrollos es significativo y permite asegurar ciertos niveles de confianza cuando otros usan un Web Browser como cliente para sus Sistemas. 

Actualmente muchos desarrollos de software crean sistemas que están conectados a la Internet, pues permite agregar funcionalidades al sistema y facilidades para sus \textit{Stakeholders}. Esto lleva a depender de un cliente web, cómo un \textit{Web Browser} que permita el acceso a los servicios, datos u operaciones que el sistema entrega. Sin embargo, la Internet influye en la superficie de ataque del nuevo sistema que se implementó, y lamentablemente tanto Stakeholders como muchos desarrolladores no están al tanto de los riesgos a los que se enfrentan.

En esta Memoria presentada al Departamento de Informática (DI) de la UTFSM\footnote{Universidad Técnica Federico Santa María} Casa Central, se al incursionará en el ámbito de la seguridad del Web Browser, con el objetivo de obtener documentos formales que servirán como herramientas a personas que Desarrollen Software y hagan un fuerte uso del Navegador para las actividades del sistema desarrollado.

\section*{Abstract}

The Web Browser is known as one of the most used applications - or \textit{killer app} - and also was the first introduced when the Internet was created (1990). Which is why, it's significant maturity level is above in comparison with other developements and can assure a certain level of \textit{trust} whenever it is used as a client with other systems.

\label{chap:abstract}


