% Resumen a grandes rasgos de lo existente (resumen del estado del arte y resumen del trabajo realizado)

\section*{Resumen}
\label{chap:resumen}

El Web Browser es una de las aplicaciones más usadas - \textit{killer app} - y también una de las primeras en aparecer en cuanto se creó el Internet (década de los 90). Por lo mismo, su nivel de madurez con respecto a otros desarrollos es significativo y permite asegurar ciertos niveles de confianza cuando otros usan un Web Browser como cliente para sus Sistemas. 

Actualmente muchos desarrollos de software crean sistemas que están conectados a la Internet, pues permite agregar funcionalidades al sistema y facilidades para sus \textit{Stakeholders}. Esto lleva a depender de un cliente web, cómo un \textit{Web Browser}, que permite el acceso a los servicios, datos u operaciones que el sistema entrega. Sin embargo, la Internet influye en la superficie de ataque del nuevo sistema, y lamentablemente tanto Stakeholders como muchos desarrolladores no están al tanto de los riesgos a los que se exponen.

Al tener sistemas que se interconectan con el Web Browser, Stakeholder como Desarrolladores deben estar al tanto de los posibles riesgos que podrían enfrentar. La falta de educación de Seguridad en los desarrolladores de software de un proyecto, la poca y dispersa documentación de cada navegador (así como su estandarización), podría llegar a ser un flanco débil en el desarrollo de grandes architecturas que dependen del Browser para realizar sus servicios. Una Arquitectura de Referencia del Web Browser, utilizando Patrones Arquitecturales, podría ser una base para el entendimiento de los mecanismos de seguridad y su Arquitectura, que interactua con un sistema Web mayor. Ésto mismo, entregaría una unificación de ideas y terminología, al dar una mirada holística sin tener en cuenta detalles de implementación tanto del Browser como el sistema con el que interactua.

En esta Memoria presentada al Departamento de Informática (DI) de la UTFSM\footnote{Universidad Técnica Federico Santa María} Casa Central, incursionará en el ámbito de la seguridad del Web Browser, tiene como objetivo el obtener documentos semi-formales que servirán como herramientas a personas que desarrollen Software y hagan un fuerte uso del Navegador para las actividades del sistema desarrollado.

\section*{Abstract}

The Web Browser is known as one of the most used applications - or \textit{killer app} - and also was the first introduced when the Internet was created (1990s). Which is why, its significant maturity level is above in comparison with other developements and can assure a certain level of \textit{trust} whenever it is used as a client with other systems.

Currently a lot of software developments create systems that are connected to the Internet, which allows to add functionality within a system and facilities to their \textit{Stakeholders}. This leads to depend in a \textit{web client}, as the \textit{Web Browser}, which allows access to services, data or operations that the system delivers. Nevertheless, the Internet influences the attack surface of the new system, and unfortunately many stakeholders and developers are not aware of the risks they are exposed.

Having systems which are interconnected with the Web Browser, Stakeholder and Developers should be aware of the potential risks they could face. The lack of Security Education in Software developers of a project, the low and scattered documentation of each browser (and standardization), could become a great flaw in big architectural developments which depends on the browser to do their services. A Reference Architecture of the Web Browser, using Architectural Patterns, could be a base for understanding the security mechanisms and its architecture, which interacts with a bigger web system. This would give an unification of ideas and terminology, giving a holistic view regardless the implementation details for both the browser and the system it communicates to.

This work presented to the Departamento de Informática (DI) of the UTFSM\footnote{Universidad Técnica Federico Santa María} Casa Central, will seek within the scope of Web Browser Security, has the objective or goal to obtain semi-formal documents that can help as tools to software developers who make a strong use of a web browser within the activities of the systems they are building.



\label{chap:abstract}


