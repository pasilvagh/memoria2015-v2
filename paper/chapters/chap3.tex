\chapter{Marco Teórico - (In) Seguridad en el Browser}
\label{chap3:FC}

En esta sección se presentan los posibles ataques que un Browser puede sufrir y que directamente podrían afectar al sistema con el que se comunica. Principalmente ahondaremos en los ataques en el Browser relacionados a las técnicas de Ingeniería Social \cite{socEngineeering}. El escenario actual de los ataques en el browser ha cambiado bastante, si es comparado a aquellos de la decada de los noventa. Cada día los Browsers son más robuztos y difíciles de explotar, por lo mismo, los ataques de tipo \textit{drive-by downloads} o los basados en ejecución de código para vulnerar el sistema, cada vez son menores. Una nueva forma de ataques ha emergido y es al mismo tiempo, una forma más fácil de lograrlo, pues se basa en el engaño del usuario a realizar lo que el atacante desea. Una vez el usuario es engañado, el atacante puede lograr un control total tanto del Browser como del Host, sin haber tenido que vulnerar el sistema \cite{Rajab2013}. Desarrollos de sistemas críticos que interactuan a diario con diferentes usuarios en la red, deberían de ser los más preocupados de estos ataques pues atentan contra la confidencialidad, integridad y disponibilidad de los datos, tanto del usuario (personales) como los de los \textit{Stakeholders} involucrados.

\section{Social Engineering}
\cite{socEngineeering} define este tipo de acción como: El acto de manipular una persona para realizar acciones que no son parte de los mejores intereses del \textit{blanco o víctima} (la misma persona/organización/etc u otra entidad). Un ataque de éste tipo puede darse de diversas maneras, no dejando la posibilidad de un encuentro físico o digital con el que realiza el engaño. Un ataque basado en ingeniería social, es uno que se aprovecha del comportamiento humano y la confianza de la víctima. En el contexto del Web Browser, el usuario engañado es la primera y última linea de defensa contra este tipo de ataques, pues un abuso en la confianza del usuario podría abrir las puertas al Host del Browser, logrando un daño tanto del usuario como con los sistemas externos con los que interactúa.

\section{Ataques y Amenazas}
